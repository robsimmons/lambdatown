\documentclass{article}

\begin{document}

\title{A dependent skeleton calculus}
\author{Salil Joshi \and Robert J. Simmons}
\maketitle

\section{Definition}

Terms, skeletons, and substitutions are assumed to be implicitly 
well-simply-typed and well-scoped. The upshot of this is essentially 
that we know from the get-go that hereditary normalization $N[M/x]$ is total 
by induction on the implicit type of the term $M$.

\[
\begin{array}{rrcl}
\mbox{Terms} & M, N & ::= & 
     x \cdot K [ \sigma ] 
\mid c \cdot K [ \sigma ]
\mid \lambda x. N
\mid \langle N_1 , N_2 \rangle 
\\
\mbox{Substitutions} & \sigma & ::= &
     \cdot 
\mid N/x, \sigma
\\
\mbox{Skeletons} & K & ::= &
     \cdot
\mid x, K
\mid \pi_1 K
\mid \pi_2 K
\medskip\\
\mbox{Types} & A,B & ::= &
     a \cdot K [ \sigma ]
\mid \Pi x{:}A.B
\mid \Sigma x{:}A.B
\\
\mbox{Kinds} & K & ::= &
     \mathsf{type} \
\mid \Pi x{:}A.K
\medskip\\
\mbox{Contexts} & \Gamma & ::= & 
     \cdot
\mid \Gamma, x : A
\\
\mbox{Partial Contexts} & \Delta & ::= &
     \cdot
\mid x : A, \Delta
\\
\mbox{Signatures} & \Sigma & ::= &
     \cdot
\mid \Sigma, a : K
\mid \Sigma, c : A
\\
\end{array}
\]

\end{document}